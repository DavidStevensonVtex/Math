% This is a small sample LaTeX input file (Version of 10 April 1994)
%
% Use this file as a model for making your own LaTeX input file.
% Everything to the right of a  %  is a remark to you and is ignored by LaTeX.

% The Local Guide tells how to run LaTeX.

% WARNING!  Do not type any of the following 10 characters except as directed:
%                &   $   #   %   _   {   }   ^   ~   \


\documentclass[twocolumn]{book}        % Your input file must contain these two lines
\usepackage[margin=1cm]{geometry} 

\begin{document}               % plus the \end{document} command at the end.

\chapter{Getting Acquainted}

\chapter{Getting Started}

\chapter{Carrying On}

Three modes
\footnote{Paragraph mode corresponds to the vertical and ordinary horizontal modes in {\em The \TeX{}book}, and 
LR mode is called restricted horizontal mode there. \LaTeX\ also has a restricted form of LR mode called 
{\em picture} mode that is described in Section 7.1.}
:

\begin{enumerate}
\item \textbf{paragraph mode} input as a sequence of words and sentences to be broken into lines, paragraphs and pages.
\item \textbf{math mode} Math mode begins with a command like \$ or \textbackslash ( or \textbackslash [ or 
\textbackslash begin\{equation\}, and leaves when finding the corresponding command that ends the formula.
\item \textbf{left-to-right mode or LR mode} LR mode consisers your input to be a string of words with spaces between them. 
It keeps going from left to right; it never starts a new line.
\end{enumerate}

\section{Changing the Type Style}

Type style is used to indicate logical structure. In this book, emphasized text appears in {\em italic} style type and 
\LaTeX{} input in {\tt typewriter} style. In \LaTeX{}, a type style is specified by three components: 
shape, series, and family.

\subsubsection*{Shapes}
\begin{itemize}
    \item Upgright shape (default).  \textbackslash textup\{Upgright shape...\}
    \item {\em Italic shape}. \textbackslash textit\{Italic shape...\}
    \item \textsl{Slanted shape}. \textbackslash textsl\{Slanted shape...\}
    \item \textsc{Small Caps Shape.} \textbackslash textsc\{Small caps shape...\}
\end{itemize}
\subsubsection*{Series}
\begin{itemize}
    \item \textmd{Medium series (default)}. \textbackslash textmd\{Medium series...\}
    \item \textbf{Boldface series}. \textbackslash textbf\{Boldface series...\}
\end{itemize}
\subsubsection*{Family}
\begin{itemize}
    \item \textrm{Roman family (default)}. \textbackslash textrm\{Roman family...\}
    \item \textsf{Sans serif family}. \textbackslash textsf\{Sans serif family....\}
    \item \texttt{Typewriter family}. \textbackslash texttt\{Typewriter family...\}
\end{itemize}

These commands can be combined in a logical fashion to produce a wide vaiety of type styles.

\textsf{\textbf{Who \textmd{on} Earth}} is \textit{\texttt{ever}} 
going to use boldface sans serif or an italic typewriter type style?

Each of the text-style commands described above has a corresponding declaration.
Boldface text can be obtained with either the {\textbf \textbackslash textbf} text-producing
command or the {\textbf \textbackslash bfseries} declaration.

\textbf{More} and {\bfseries more} armadillos are crossing the road.

The declarations corresponding to the text-producing commands are:

\begin{itemize}
    \item \textbf{cmd} \textbf{decl}
    \item \textbackslash textup \textbackslash upshape
    \item \textbackslash textit \textbackslash itshape
    \item \textbackslash textsl \textbackslash slshape
    \item \textbackslash textsc \textbackslash scshape
    \item \textbackslash textmd \textbackslash upshape
    \item \textbackslash textbf \textbackslash upshape
    \item \textbackslash textrm \textbackslash upshape
    \item \textbackslash textsf \textbackslash upshape
    \item \textbackslash texttt \textbackslash upshape
    \item \textbackslash textup \textbackslash upshape
\end{itemize}

None of test text-producing commands or declarations can be used in math mode.
Section 3.3.8 explains how to change type style in a mathematical formula.

Type style is a visual property.
Commands to specify visual properties belong not in the text, but in the 
definitions of commands that describe logical structure.
\LaTeX\ provides the \textbf{\textbackslash emph} command for emphaiszed text;
Section 3.4 explains how to define your own commands for the logical structure in your document.

\section{Symbols from Other Languages}

The \textbf{\tt babel} package allows you to produce documents in languages other than English,
as well as multilanguage documents.



\end{document}                 % The input file ends with this command.
