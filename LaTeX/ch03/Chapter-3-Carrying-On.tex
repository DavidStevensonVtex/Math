\documentclass[twocolumn]{book}        % Your input file must contain these two lines
\usepackage[margin=1cm]{geometry} 
\usepackage[T1]{fontenc}
\usepackage{lmodern}
\usepackage{textcomp}
\usepackage{latexsym}



\begin{document}               % plus the \end{document} command at the end.

\chapter{Getting Acquainted}

\chapter{Getting Started}

\chapter{Carrying On}

Three modes
\footnote{Paragraph mode corresponds to the vertical and ordinary horizontal modes in {\em The \TeX{}book}, and 
LR mode is called restricted horizontal mode there. \LaTeX\ also has a restricted form of LR mode called 
{\em picture} mode that is described in Section 7.1.}
:

\begin{enumerate}
\item \textbf{paragraph mode} input as a sequence of words and sentences to be broken into lines, paragraphs and pages.
\item \textbf{math mode} Math mode begins with a command like \$ or \textbackslash ( or \textbackslash [ or 
\textbackslash begin\{equation\}, and leaves when finding the corresponding command that ends the formula.
\item \textbf{left-to-right mode or LR mode} LR mode consisers your input to be a string of words with spaces between them. 
It keeps going from left to right; it never starts a new line.
\end{enumerate}

\section{Changing the Type Style}

Type style is used to indicate logical structure. In this book, emphasized text appears in {\em italic} style type and 
\LaTeX{} input in {\tt typewriter} style. In \LaTeX{}, a type style is specified by three components: 
shape, series, and family.

\subsubsection*{Shapes}
\begin{itemize}
    \item Upgright shape (default).  \textbackslash textup\{Upgright shape...\}
    \item {\em Italic shape}. \textbackslash textit\{Italic shape...\}
    \item \textsl{Slanted shape}. \textbackslash textsl\{Slanted shape...\}
    \item \textsc{Small Caps Shape.} \textbackslash textsc\{Small caps shape...\}
\end{itemize}
\subsubsection*{Series}
\begin{itemize}
    \item \textmd{Medium series (default)}. \textbackslash textmd\{Medium series...\}
    \item \textbf{Boldface series}. \textbackslash textbf\{Boldface series...\}
\end{itemize}
\subsubsection*{Family}
\begin{itemize}
    \item \textrm{Roman family (default)}. \textbackslash textrm\{Roman family...\}
    \item \textsf{Sans serif family}. \textbackslash textsf\{Sans serif family....\}
    \item \texttt{Typewriter family}. \textbackslash texttt\{Typewriter family...\}
\end{itemize}

These commands can be combined in a logical fashion to produce a wide vaiety of type styles.

\textsf{\textbf{Who \textmd{on} Earth}} is \textit{\texttt{ever}} 
going to use boldface sans serif or an italic typewriter type style?

Each of the text-style commands described above has a corresponding declaration.
Boldface text can be obtained with either the {\textbf \textbackslash textbf} text-producing
command or the {\textbf \textbackslash bfseries} declaration.

\textbf{More} and {\bfseries more} armadillos are crossing the road.

The declarations corresponding to the text-producing commands are:

\begin{itemize}
    \item \textbf{cmd} \textbf{decl}
    \item \textbackslash textup \textbackslash upshape
    \item \textbackslash textit \textbackslash itshape
    \item \textbackslash textsl \textbackslash slshape
    \item \textbackslash textsc \textbackslash scshape
    \item \textbackslash textmd \textbackslash upshape
    \item \textbackslash textbf \textbackslash upshape
    \item \textbackslash textrm \textbackslash upshape
    \item \textbackslash textsf \textbackslash upshape
    \item \textbackslash texttt \textbackslash upshape
    \item \textbackslash textup \textbackslash upshape
\end{itemize}

None of test text-producing commands or declarations can be used in math mode.
Section 3.3.8 explains how to change type style in a mathematical formula.

Type style is a visual property.
Commands to specify visual properties belong not in the text, but in the 
definitions of commands that describe logical structure.
\LaTeX\ provides the \textbf{\textbackslash emph} command for emphaiszed text;
Section 3.4 explains how to define your own commands for the logical structure in your document.

\section{Symbols from Other Languages}

The \textbf{\tt babel} package allows you to produce documents in languages other than English,
as well as multilanguage documents.

\subsection{Accents}

Note: While \LaTeX\ accents annotations work, .tex files also support Unicode.
This file is UTF-8.

El se\~{n}or est\'{a} bien, gar\c{c}on.

El señor está bien, garçon.

The letters $i$ and $j$ need special treatment because they should lose their dots when accented.
The commands \textbackslash i and \textbackslash j produce a dotless $i$ and $j$, respectively.

\'{E}l est\'{a} aqu\'{\i}.

\subsection{Symbols}

The commands in Table 3.2 can appear only in paragraph and LR modes; use an \mbox command to 
put one inside a mathematical formula.

The following six special punctuation symbosl can be used in amy mode:

\begin{itemize}
    \item \dag \textbackslash dag 
    \item \ddag \textbackslash ddag
    \item \S \textbackslash S
    \item \P \textbackslash P 
    \item \copyright \textbackslash copyright
    \item \pounds \textbackslash pounds
\end{itemize}

\section{Mathematical Formulas}

A formula that appears in the running text, called an {\em in-text} formula, is produced by 
the {\texttt{\textbf{math}}} environment. This environment can be invoked with either of the
two shortforms \textbackslash ( ... \textbackslash ) or \$...\$ as well as by the usual 
\textbackslash {\bf begin} ... \textbackslash {\bf end} construction.

The {\texttt{\textbf displaymath}} enviornment, which has the short form 
\textbackslash [...\textbackslash ], produces an unnumbered displayed formula.
The short forms \$...\$, \textbackslash (...\textbackslash ), and 
\textbackslash [...\textbackslash ] act as full-fledged environments.
A numbered displayed formula is produced by the \textbf{equation} environment.
Section 4.2 describes commands for assigning names to equation numbers and referring
to the numbers by name, so you don't have to keep track of the actual numbers.

The \texttt{\textbf math}, \texttt{\textbf{displaymath}}, and \texttt{\textbf{equation}}
environments put \TeX\ in math mode. \TeX\ ignores spaces in the input when it's in math mode
(but space characters ma still be needed to mark the end of a command name).
Section 3.3.7 describes how to add and remove space in formulas. Remember that
\TeX\ is in LR mode, where spaces in the input generate space in the output,
when it begins processing the argument of an \textbackslash{\tt mbox} command--even 
one that appears inside a formula.

All the commands introduced in this section can be used only in math mode,
unless it is explicitly stated that they can be used elsewhere. Except as noted, they are all robust.
However, \textbackslash{\tt begin}, \textbackslash{\tt end}, \textbackslash (, \textbackslash ),
\textbackslash [, and \textbackslash ] are fragile commands.

\subsection{Some Common Structures}

\subsubsection*{Subscripts and Superscripts}

Subscripts and superscripts are made with the \_ and \^\ commands.

\begin{itemize}
    \item $x^{2y}$ -- $x$\char `^\{$2y$\}
    \item $x_{2y}$ -- $x$\_\{$2y$\}
    \item $x^{y^2}$
    \item $x^{y_1}$
    \item $x^{y}_{1}$
    \item $x_{1}^{y}$
\end{itemize}

\subsubsection*{Fractions}

Fractions denoted by the / symbol are made in the obvious way.

Multiplying by $n/2$ gives \( (m+n)/n \).

Most fractions in the running text are written this way. The \textbackslash{\tt frac} command 
is used for large fractions in displayed formulas; it has two arguments:
the numerator and the denominator.

\[ x = \frac{y+z/2}{y^{2}+1} \]

\[ x = \frac{(y+z)/2}{y^{2}+1} \]

\[ \frac{x+y}{1 + \frac{y}{z+1}} \]

The \textbackslash{\tt frac} command can be used in an in-text formula to produce a fraction like
$\frac{1}{2}$ (by typing \$\textbackslash{frac\{1\}\{2\}}), but this is seldom done.

\subsubsection*{Roots}

The \textbackslash sqrt command produces the square root of its argument; it has an optional first 
argument for other roots. It is a fragile command.

A square root $\sqrt{x+y}$ and an \emph{n}th root $\sqrt[n]{2}$.

\subsubsection{Ellipsis}

The commands \texttt{\textbackslash ldots} between commas produce two different kinds of ellipsis.

A low ellipsis: $x_{1}, \ldots ,x_{n}$.

A centered ellipsis: $a + \cdots + z$.

Use \texttt{\textbackslash ldots} between commas and between juxtaposed symbols like $a\ldots z$; 
use \texttt{\textbackslash cdots} between symbols like +, -, and =. \TeX\ can also produce verticaland
diagonal ellipsis, which are used mainly in arrays.

$\vdots$ \texttt{\textbackslash vdots}

$\ddots$ \texttt{\textbackslash ddots}

\subsection{Mathematical Symbols}

Remember that mathematical symbols can be used only in math mode.

\subsubsection{Greek Letters}

The command to produce a lowercase Green letter is obtained by adding a \textbackslash to
the name of the letter. For an uppercase Greek letter, just capitalize the first letter of 
the command name.

Making Greek letters is as easy as $\pi$ (or $\Pi$).

If the uppercase Greek letter is the same as its Roman equivalent, as in uppercase alpha,
then there is no command to generate it. A complete list of commands for making Greek
letters appears in Table 3.3. Note that some of the lowercase letters may have variant forms,
made by commands beginning with  \texttt{\textbackslash var}. Also, observe that there's no
special command for an omicron, you just use an \texttt{o}.

\emph{Lowercase}

\begin{itemize}
    \item $\alpha$ \texttt{\textbackslash alpha}
    \item $\beta$ \texttt{\textbackslash beta}
    \item $\gamma$ \texttt{\textbackslash gamma}
    \item $\delta$ \texttt{\textbackslash delta}
    \item $\epsilon$ \texttt{\textbackslash epsilon}
    \item $\varepsilon$ \texttt{\textbackslash varepsilon}
    \item $\eta$ \texttt{\textbackslash eta}
    \item $\theta$ \texttt{\textbackslash theta}
    \item $\vartheta$ \texttt{\textbackslash vartheta}
    \item $\iota$ \texttt{\textbackslash iota}
    \item $\kappa$ \texttt{\textbackslash kappa}
    \item $\mu$ \texttt{\textbackslash mu}
    \item $\nu$ \texttt{\textbackslash nu}
    \item $\xi$ \texttt{\textbackslash xi}
    \item $o$ \texttt{o}
    \item $\pi$ \texttt{\textbackslash pi}
    \item $\varpi$ \texttt{\textbackslash varpi}
    \item $\rho$ \texttt{\textbackslash rho}
    \item $\varrho$ \texttt{\textbackslash varrho}
    \item $\sigma$ \texttt{\textbackslash sigma}
    \item $\varsigma$ \texttt{\textbackslash varsigma}
    \item $\tau$ \texttt{\textbackslash tau}
    \item $\upsilon$ \texttt{\textbackslash upsilon}
    \item $\phi$ \texttt{\textbackslash phi}
    \item $\varphi$ \texttt{\textbackslash varphi}
    \item $\chi$ \texttt{\textbackslash chi}
    \item $\psi$ \texttt{\textbackslash psi}
    \item $\omega$ \texttt{\textbackslash omega}
\end{itemize}

\emph{Uppercase}

\begin{itemize}
    \item ${\Gamma}$ \texttt{\textbackslash Gamma}
    \item ${\Delta}$ \texttt{\textbackslash Delta}
    \item ${\Theta}$ \texttt{\textbackslash Theta}
    \item ${\Lambda}$ \texttt{\textbackslash Lambda}
    \item ${\Xi}$ \texttt{\textbackslash Xi}
    \item ${\Pi}$ \texttt{\textbackslash Pi}
    \item ${\Sigma}$ \texttt{\textbackslash Sigma}
    \item ${\Upsilon}$ \texttt{\textbackslash Upsilon}
    \item ${\Phi}$ \texttt{\textbackslash Phi}
    \item ${\Psi}$ \texttt{\textbackslash Psi}
    \item ${\Omega}$ \texttt{\textbackslash Omega}
\end{itemize}

\subsubsection{Calligraphic Letters}

\TeX\ provides twenty-size calligraphic letters 
$\mathcal{A}, \mathcal{B}, \ldots, \mathcal{Z}$, also called script letters.
They are provided by a special type style invoked with the \texttt{\textbackslash mathcal}
command.

The shaded symbols require the \texttt{latexsym} package to be loaded with 
a \texttt{\textbackslash usepackage} command.

Additional symbols can be made by stacking one symbol on top of another with the 
\texttt{\textbackslash stackrel} command of Section 3.3.6 or the array environment 
of Section 3.3.3.

If $x \not< y$ then \( x \not\leq y-1\).

\emph{Binary Operation Symbols}


\begin{itemize}
    \item $\pm$ \texttt{\textbackslash pm}
    \item $\mp$ \texttt{\textbackslash mp}
    \item $\times$ \texttt{\textbackslash times}
    \item $\div$ \texttt{\textbackslash div}
    \item $\ast$ \texttt{\textbackslash ast}
    \item $\star$ \texttt{\textbackslash star}
    \item $\circ$ \texttt{\textbackslash circ}
    \item $\bullet$ \texttt{\textbackslash bullet}
    \item $\cdot$ \texttt{\textbackslash cdot}
    \item $\cap$ \texttt{\textbackslash cap}
    \item $\cup$ \texttt{\textbackslash cup}
    \item $\uplus$ \texttt{\textbackslash uplus}
    \item $\sqcap$ \texttt{\textbackslash sqcap}
    \item $\sqcup$ \texttt{\textbackslash sqcup}
    \item $\vee$ \texttt{\textbackslash vee}
    \item $\wedge$ \texttt{\textbackslash wedge}
    \item $\setminus$ \texttt{\textbackslash setminus}
    \item $\wr$ \texttt{\textbackslash wr}
    \item $\diamond$ \texttt{\textbackslash diamond}
    \item $\bigtriangleup$ \texttt{\textbackslash bigtriangleup}
    \item $\bigtriangledown$ \texttt{\textbackslash bigtriangledown}
    \item $\triangleleft$ \texttt{\textbackslash triangleleft}
    \item $\triangleright$ \texttt{\textbackslash triangleright}
    \item $\lhd$ \texttt{\textbackslash lhd}
    \item $\rhd$ \texttt{\textbackslash rhd}
    \item $\unlhd$ \texttt{\textbackslash unlhd}
    \item $\unrhd$ \texttt{\textbackslash unrhd}
    \item $\oplus$ \texttt{\textbackslash oplus}
    \item $\ominus$ \texttt{\textbackslash ominus}
    \item $\otimes$ \texttt{\textbackslash otimes}
    \item $\oslash$ \texttt{\textbackslash oslash}
    \item $\odot$ \texttt{\textbackslash odot}
    \item $\bigcirc$ \texttt{\textbackslash bigcirc}
    \item $\dagger$ \texttt{\textbackslash dagger}
    \item $\ddagger$ \texttt{\textbackslash ddagger}
    \item $\amalg$ \texttt{\textbackslash amalg}
\end{itemize}

Here's how they look when displayed:
\[\sum_{i=1}^{n} x_{i} = \int_{0}^{1}f \]
and in the text:
\(\sum_{i=1}^{n} x_{i} = \int_{0}^{1}f \)

Section 3.3.8 tells how to coerce \TeX\ into producing \(\sum_{i=1}^{n}\) 
in a displayed formula and \(\sum_{i=1}^{n}\)  in an in-text formula.

\subsubsection{Log-like Functions}

Logarithms obey the law: \( \log xy = \log x + \log y \).

\( \gcd(m,n) = a \bmod b \) 

\( x \equiv y \pmod{a+b} \)

Note that \texttt{\textbackslash pmod} has an argument and produces parentheses, while 
\texttt{\textbackslash bmod} produces only the ``mod''.

Some log-like functions act the same as the variable-sized symbols of Table 3.8 with 
respect to subscripts.

As a displayed formula:
\[ \lim_{n \rightarrow \infty} x = 0 \]
but in text:
\( \lim_{n \rightarrow \infty} x = 0 \) 

\end{document}
