% This is a small sample LaTeX input file (Version of 10 April 1994)
%
% Use this file as a model for making your own LaTeX input file.
% Everything to the right of a  %  is a remark to you and is ignored by LaTeX.

% The Local Guide tells how to run LaTeX.

% WARNING!  Do not type any of the following 10 characters except as directed:
%                &   $   #   %   _   {   }   ^   ~   \


\documentclass[twocolumn]{article}        % Your input file must contain these two lines
\usepackage[margin=1cm]{geometry} 

\begin{document}               % plus the \end{document} command at the end.

\section{Chapter 1: Getting Acquainted}

\section{Chapter 2: Getting Started}

\subsection{Preparing an Input File}

\subsection{The Input}

\subsubsection{Sentences and Paragraphs}

Words are separated by one or more spaces.  Paragraphs are separated by
one or more blank lines.  The output is not affected by adding extra
spaces or extra blank lines to the input file.


\subsubsection*{A Warning or Two}  % This command makes a subsection title.

If you get too much space after a mid-sentence period---abbreviations
like etc.\ are the common culprits)---then type a backslash followed by
a space after the period, as in this sentence.

\subsubsection*{Quotation Marks}

Double quotes are typed like this: ``quoted text''.
Single quotes are typed like this: `single-quoted text'.

\subsubsection*{Dashes}

You can produce three different sizes of dash by typing one, two, or three ``-'' characters:

Intra-word dash: X-ray

Medium dash: 1--2

A punctuation dash---like this.

Long dashes are typed as three dash characters---like this.

\subsubsection*{Space After a Period}

You tell TeX that a period doesn't end a sentence by using the backslash space command (a backslash followed by a space or end of line).

Tinker et al.\ made the double play.

Don't leave any space between the period and the backslash.

\subsubsection*{Special Symbols}

Remember, don't type the 10 special characters (such as dollar sign and
backslash) except as directed!  The following seven are printed by
typing a backslash in front of them:  \$  \&  \#  \%  \_  \{  and  \}.
The manual tells how to make other symbols.

\subsubsection*{Simple-Text-Generating Commands}

Some people use plain \TeX, but I prefer \LaTeX.

This page was produced on \today.

The following is an ellipsis example.

If nominated \ldots, I will not serve.

\subsubsection*{Emphasizing Text}

Emphasized text is typed like this: \emph{this is emphasized}.
Bold       text is typed like this: \textbf{this is bold}.

You can have \emph{empahsized text \emph{within} emphasized text} too.

\subsubsection*{Preventing Line Breaks}

Typesetters break lines between words when possible and split words only between syllables 
(inserting a hyphen at the break).

Sometimes a line should not be broken between or within words. Human typesetters recognize 
these situations, but \TeX bust be told about some of them.

Line breaking should be prevented at certain interword spaces.

Typing a $\sim$ (a tilde character) produces ordinary interword space at which \TeX\ will never break a line.

Below are some examples indicating when a $\sim$ should be used.

Mr.~Jones

Figure~7

(1)~gnats

U.~S.~Grant

from 1 to~10

The \textbackslash  mbox command tells \TeX\ to print its entire argument on the same line.

Doctor \mbox{Lamport}, I presume?

\subsubsection*{Footnotes}

Footnotes are produced with the footnote command having the text of the footnote as its argument.

Gnus\footnote{A gnu is a big animal.} can be quite a gnusance.

A \textbackslash footnote command cannot be used in the argument of most commands;
for example, it can't appear in the argument of an \textbackslash footnote command.

\subsubsection*{Formulas}

If you're writing a technical document, it's likely to contain mathematical formulas.
A formula appearing in the middle of a sentence is enclosed by \textbackslash ( 
and \textbackslash ) commands.

Any spaces that you type in the formula are ignored.

Does \( x + y \) always equal \(y+x\)?

\TeX\ regards a formula as a word, which may be broken across lines at certain points, 
and space before the \textbackslash (  or after the \textbackslash ) is treated as an 
ordinary interword separation.

Subscripts are produced by the \_ command and superscripts by the \^{} command.

\( a_{1} > x^{2n} / y^{2n} \)

These two commands can only be used inside a mathematical formula.

When used in a formula, the right-quote character ' produces a prime (\('\)),
two in a row produce a double prime.

\break
\( x' < x'' - y'_{3} < 10 x''' z\).

The formula \( a<7 \) is a noun in this sentence.
It is sometimes used as a clause by writing that \( a<7 \).

Beginning a sentence with a formula makes it hard to find the start of the 
sentence; don't do it.

A variable like \( x \) is a formula. To save you some typing, \LaTeX\ treats
\$...\$ the same as \textbackslash (...\textbackslash ).

Let $x$ be a prime such that $y>2x$.

Use \$...\$ only for a short formula, such as a single variable. 
It's easy to forget one of the \$ characters that surrounds along formula.
You can also type

\textbackslash begin\{math\} ... \textbackslash end\{math\}

\noindent
instead of \textbackslash ( ... \textbackslash (. You might want to use this form 
for a very long formula.

\subsubsection*{Ignorable Input}

When \TeX\ encounters a \% character in the input, it ignores the \% and all 
characters following it on that line -- including the space charactr that ends the line.
\TeX\ also ignores spaces at the beginning of the next line.

Gnus and armadi% More @_#!$^{& gnus?
llos are generally tolerant of one another and seldom quarrel.

The \% has two uses: ending a line without producing any space in the output
]footnote{However, you can't split a command across two lines.}
and putting a comment (a note to yourself) in the input file.

\subsubsection{The Document}

The text of every document starts with \textbackslash begin\{document\} command
and ends with an \textbackslash end\{document\}. The part of the input file preceding
the \textbackslash begin\{document\} command is called the \emph{preamble}.

\subsubsection*{The Document Class}

The preamble begins 
\footnote{As explained in Section 4.7, the \textbackslash documentclass 
command may actually be preceeded by prepended files.}
with a \textbackslash documentclass command whose argument is one of the
predefined classes of document that \LaTeX\ knows about. The file sample2e.tex
begings with

\textbackslash documentclass\{article\}

\noindent
which selects the article class. The other standard \LaTeX\ classs used for ordinary
documents is the report class. The article class is generally used for shorter
documents than the report class. Other standard classes are described in Chapter 5.

In addiction to choosing the class, you can also select from among certain document-class options.
The options for the article and report classes include the following.

\hfill \break
\emph{11pt}    Specifies a size of type known as \emph{eleven point}, which is ten percent larger 
than the ten-point type normally used.

\hfill \break
\noindent
\emph{12pt}    Specifies a twelve-point type size, which is twenty percent larger than ten point.

\hfill \break
\emph{twoside} Formats the output for printing on both sides of the page. (\LaTeX\ has no control over the actual printing.)

\hfill \break
\emph{twocolumn} Produces two-column output.

You specify a document-class option by enclosing it in square brackets immediately after 
the ``\textbackslash documentclass'', as in


\textbackslash documentclass[twoside]\{report\}

\noindent
Multiple options are separated by commas.

\textbackslash documentclass [twocolumn,12pt]\{article\}

Don't leave any space inside the square brackets.

The \textbackslash documentclass command can be used either with or without the option-choosing part.
The options, enclosed in square brackets are an \emph{optional argument} of the command. It is a \LaTeX\ 
convention that optional arguments are enclosed in square brackets, while mandatory arguments are enclosed
in curly braces. \TeX\ ignores spaces after a command like \textbackslash documentclass and 
between command arguments.

The document class defines commands for specifying \LaTeX\'s standard logical structures. Additional structures 
are defined by \emph{packages}, which are loaded by the \textbackslash usepackage command. For example, the command 

\textbackslash usepackage\{latexsym\}

\noindent 
loads the latexsym package, which defines commands to produce certain special math symbols. (See Section 3.3.2).
A package can have options, specified by an optional argument of \textbackslash usepackage just like the one for 
\textbackslash documentclass.

You will probably want to define some new commands for the special structures used in your particular document.
For example, if you're writing a cookbook you will probably define your own commands for formatting recipes, as
explained in Section 3.4. These definitions should go in the preamble, after the \textbackslash documentclass and 
\textbackslash commands. The preamble can also contain commands to change some aspects of the formatting.
If you have commands or format changes that you use in several documents, you may want to define your own package,
as described in Section 6.1.4.

\subsubsection*{The Title ``Page''}

A document usually has a title ``page'' listing its title, one or more authors, and a date. I write ``page'' in quotes, 
because, for a short document, this information may be listed on the first page of text rather than on its own page.
The title information consits of the title itself, the author(s), and the date, it is specified by the three declarations
\textbackslash title, \textbackslash author, and \textbackslash date. The actual title ``page'' is generated by a 
\textbackslash maketitle command.

\textbackslash title\{Gnus of the World\}

\textbackslash author\{R. Dather \textbackslash and J. Pennings \textbackslash and B. Talkmore\}

\textbackslash date\{4 July 1997\}

\textbackslash maketitle

Note how multiple authors are separated by \textbackslash and commands.

The \textbackslash command comes after the \textbackslash begin\{document\}, usually before any other text. 
The \textbackslash title, \textbackslash author, and \textbackslash date commands can come anywhere before
the \textbackslash maketitle. The \textbackslash date is optional; \LaTeX\ supplies the current date if 
the declaration is omitted, but the \textbackslash title and \textbackslash author must appear if a 
\textbackslash make title command is used. 

Commands for adding other information, such as the author's address and an acknowledgement of support, are
described in Section C.5.4.

\subsubsection{Sectioning}

Sentences are organized into paragraphs, and paragraphs are in turn organized into a hierarchical 
\emph{section structure}. You are currently reading Subsection 2.2.3, titled \emph{Sectioning},
which is part of Section 2.2, titled \emph{The Input}, which in turn is part of Chapter 2, titled 
\emph{Getting Started}. I will use the term \emph{sectional units} for things like chapters, sections,
and subsections.

A sectional unit is begun by a sectioning command with the unit's title as its argument.

\hfill \break
\textbackslash subsection\{A Sectioning Command\}

produces

\textbf{4.7 A Sectioning Command}

\LaTeX\ automatically generates the section number. Blank lnes before or after a sectioning command have no effect.

The document class determines what sectioning commands are provided; the standard classes have the following ones.
\footnote{The names \textbackslash paragraph and \textbackslash subparagraph are unfortunate, since they denote units
that are often composed of several paragraphs; they have been retained for historical reasons.}

\textbackslash part 

\textbackslash subsection 

\textbackslash paragraph

\textbackslash chapter 

\textbackslash subsubsection 

\textbackslash subparagraph

\textbackslash section 

The article document class does not contain a \textbackslash chapter command, which makes it easy to include an 
``article'' as a chapter of a ``report''. The example above, like most others in this book, assumes the article
document class, the 4.7 indicating that this is the seventh subsection of Section 4. In the report class, 
this subsection might be numbered 5.4.7, with the ``5'' being the chapter number.

The \textbackslash part command is used for major divisions of long documents; it does not affect the numbering
of smaller units -- in the article class, if the last section of Part 1 is Section 5, then the first section of
Part 2 is Section 6.

If there is an appendix, it is begun with an \textbackslash appendix command uses the same sectioning commands
as the main part of the document. The \textbackslash appendix command does not produce any text; it simply causes
sectional units to be numbered properly for an appendix.

The document class determines the appearance of the section title, including whether or not it is numbered. 
Declarations to control section numbering are described in Section C.4, which also tells you how to make a 
table of contents.

The argument of a sectioning command may be used for more than just producing the section title; it can generate 
a table of contents entry and a running head at the top of the page. (Running heads are discussed in Section 6.1.2.)

Fragile commands are rarely used in the argument of a sectioning command. Of the commands introduced so far, the only
fragile ones are \textbackslash (, \textbackslash ),  \textbackslash begin, \textbackslash end,  and 
\textbackslash footnote -- none of which you're likely to need in a section title.
\footnote{Section C.3.3 tells you how to footnote a section title.}
On the rare occasions when you have to put a fragile command in a section title, you simply protect it with a 
\textbackslash protect command.
The \textbackslash protect command goes right before every fragile command's name, as in:

\textbackslash subsection \{Is \textbackslash protect\textbackslash (  x+y   \textbackslash protect\textbackslash ) Prime?\}

This is actually a silly example, because \$ is not a fragile command, so you can instead type

\textbackslash subsection \{Is \textbackslash \$x+y\$ Prime?\}

/noindent
but because the problem is so rare, it's hard to find a good example using the commands described in this chapter.

An argument in which fragile commands need \textbackslash protect will be called a \emph{moving} argument.
Commands that are not fragile will be called \emph{robust}.

\subsubsection{Displayed Material}

The following is an example of a short displayed quotation.

\begin{quote}
\ldots\ it's a good idea to make your input file as easy to read as possible.
\end{quote}

It is indented at both margins.


This example illustrates a type of \LaTeX\ construction called an \emph{environment}, which is typed

\begin{quote}
\textbackslash begin\{\emph{name}\} ... \textbackslash end\{\emph{name}\}
\end{quote}

where \emph{name} is the name of the environment.
The \textbf{quote} environment produces a display suitable for a short quotation.
You've already encountered two other examples of environments: the \textbf{math} environment and the 
\textbf{document} environment.

The standard \LaTeX\ document classes provide environments for producing several different kinds of displays.
Blank lines before or after the environment mark a new paragraph. Thus, a blank line after the \textbackslash end 
command means that the following text starts a new paragraph. Blank lines before and after the environment 
mean that it is a complete paragraph.

It's a bad idea to start a paragraph with displayed material, so you should not have a blank line before
a display environment without a blank line after it. Blank lines immediately following a display environment's 
\textbackslash begin command and emmediately preceding its \textbackslash end command are ignored.

\subsubsection*{Quotations}

\LaTeX\ provides two different environments for displaying quotations. The quote environment is used for 
either a short quotation or a sequence of short quotations separated by blank lines.

Our presidents have been known for their pithy remarks.
\begin{quote}
The buck stops here. \emph{Harry Truman}

I am not a crook. \emph{Richard Nixon}

It's no exaggeration to say the undecideds could go one way or another. \emph{George Bush}
\end{quote}

The \textbf{quotation} environment is used for quotations of more than one paragraph as usual, 
the paragraphs are separated by blank lines.

Here is some advice to remember when you are using \LaTeX. 
\begin{quotation}
Environments for making quotations can be used for other things as well.

Many problems can be solved by novel applications of existing environments.
\end{quotation}

\subsubsection*{Lists}

\LaTeX\ provides three list-making environments: 
\textbf{itemize}, \textbf{enumerate} and \textbf{description}.
In all three, each new list item is begun with an \textbackslash item command. 
Itemized lists are made with the
\textbf{itemize} environment and enumerated lists with the \textbf{enumerate} environment.

\begin{itemize}
    \item Each lst item is marked with a \emph{label}. The labels in this itemized list are bullets.
    \item Lists can be nested within one another.
    \begin{enumerate}
        \item The item labels in an enumerated list are numerals or letters.
        \item A list should have a least two items.
    \end{enumerate}
    \LaTeX\ permits at least four levels of nested lists, which is more than enough.

    \item Blank lnes before an item have no effect.
\end{itemize}

In the \textbf{description} environment, you specify the item labels with an optional 
argument to the \textbackslash item command, enclosed in brackets. (Although the argument
is optional, the item will look funny if you omit it.)

Three animals you should know about are:
\begin{description}
    \item[gnat] A small animal, found in the North Woods, that causes no end of trouble.
    \item[gnu] A large animal, found in crossword puzzles, that causes no end of trouble.
    \item[armadillo] a medium-sized animal, named after a medium sized Texas city. 
\end{description}

The characters [ and ] are used both to delimit and optional argument and to produce square
brackets in the output. This can cause some confusion if the text of an item begins with a [ 
or if an \textbackslash item command's optional argument contains a square bracket. 
Section C.1.1 explains what to do in these uncommon situations. All commands that have an 
optional argument are fragile.

\subsubsection*{Poetry}

Poetry is displayed with the \textbf{verse} environment. A new stanza is begun with 
one or more blank lines; lines within the stanza are separated by the 
\textbackslash \textbackslash\ command.

\begin{verse}
    There is an environment for verse \\
    Whose features some poets will curse.

    For isntead of making\\
    Them do {\em all\/} line breaking, \\
    It allows them to put too many words on a line 
    when they'd rather be forced to be terse.
\end{verse}

The \textbackslash \textbackslash * command is the same as \textbackslash \textbackslash
except that it prevents \LaTeX\ from starting a new page at that point. It can be used to
prevent a poem from being broken across pages in a distracting way. The commands
\textbackslash \textbackslash and \textbackslash \textbackslash * are used in all
environments in which you tell \LaTeX\ where to break lines; several such environments 
are described in the next chapter. The \textbackslash \textbackslash * command is called 
the *-form of the \textbackslash \textbackslash\ command. 

Several other commands also have *-forms--versions of the command that are slightly 
different from the ordinary one--that are obtained by typing * after the command name.

The \textbackslash \textbackslash and \textbackslash \textbackslash * commands have a little-used 
optional argument described in section C.1.6, so putting a [ after them presents the same probelm as 
for the \textbackslash item command. Moreover, the * in the \textbackslash \textbackslash * command 
is somewhat like an optional argument for the \textbackslash textbackslash command, so following a 
\textbackslash \textbackslash with a * in the text posees a similar problem. See Section C.1.1 for
the solutions to these unlikely problems. Almost every command that has a *-form is fragile, and its 
*-form is also fragile.

\subsubsection*{Displayed Formulas}

A mathematical formula is displayed when either it is too long to fit comfortably in the running text, 
it is so important that you want it to stand out, or it is to be numbered for future reference.

\LaTeX\ provides the \textbf{displaymath} and \textbf{equation} environments for displaying formulas;
they are the same except that \textbf{equation} numers the formula and \textbf{displaymath} doesn't.

Because displayed equations are used so frequently in mathematics, \LaTeX\ allows you to type 
\textbackslash [ ... \textbackslash ] instead of 

\begin{quote}
    \textbackslash begin\{displaymath\} ... \textbackslash end\{displaymath\}
\end{quote}

Here is an example of an unnumbered displayed equation:
\[ x' + y^{2} = z_{i}^{2} \] 
and here is the same equation numbered:
\begin{equation}
    x' + y^{2} = z_{i}^{2}
\end{equation}

The document class determines how equations are numbered. Section 4.2 describes how \LaTeX\ 
can automatically handle references to equation numbers so you don't have to keep track of
the numbers.

A displayed formula, like any displayed text, should not begin with a paragraph. Moreover, 
it should not form a complete paragraph by itself. These two observations are summed up in a
simple rule: in the input, never leave a blank line before a displayed formula.

\TeX\ will not break the formula is a \textbf{displaymath} or \textbf{equation} environment across lines.
See Section 3.3.5 for commands to create a single multiple-line formula or a sequence of displayed formulas.


\end{document}                 % The input file ends with this command.
