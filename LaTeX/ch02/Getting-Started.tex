% This is a small sample LaTeX input file (Version of 10 April 1994)
%
% Use this file as a model for making your own LaTeX input file.
% Everything to the right of a  %  is a remark to you and is ignored by LaTeX.

% The Local Guide tells how to run LaTeX.

% WARNING!  Do not type any of the following 10 characters except as directed:
%                &   $   #   %   _   {   }   ^   ~   \


\documentclass[twocolumn]{article}        % Your input file must contain these two lines
\usepackage[margin=1cm]{geometry} 

\begin{document}               % plus the \end{document} command at the end.

\section{Chapter 1: Getting Acquainted}

\section{Chapter 2: Getting Started}

\subsection{Preparing an Input File}

\subsection{The Input}

\subsubsection{Sentences and Paragraphs}

Words are separated by one or more spaces.  Paragraphs are separated by
one or more blank lines.  The output is not affected by adding extra
spaces or extra blank lines to the input file.


\subsubsection*{A Warning or Two}  % This command makes a subsection title.

If you get too much space after a mid-sentence period---abbreviations
like etc.\ are the common culprits)---then type a backslash followed by
a space after the period, as in this sentence.

\subsubsection*{Quotation Marks}

Double quotes are typed like this: ``quoted text''.
Single quotes are typed like this: `single-quoted text'.

\subsubsection*{Dashes}

You can produce three different sizes of dash by typing one, two, or three ``-'' characters:

Intra-word dash: X-ray

Medium dash: 1--2

A punctuation dash---like this.

Long dashes are typed as three dash characters---like this.

\subsubsection*{Space After a Period}

You tell TeX that a period doesn't end a sentence by using the backslash space command (a backslash followed by a space or end of line).

Tinker et al.\ made the double play.

Don't leave any space between the period and the backslash.

\subsubsection*{Special Symbols}

Remember, don't type the 10 special characters (such as dollar sign and
backslash) except as directed!  The following seven are printed by
typing a backslash in front of them:  \$  \&  \#  \%  \_  \{  and  \}.
The manual tells how to make other symbols.

\subsubsection*{Simple-Text-Generating Commands}

Some people use plain \TeX, but I prefer \LaTeX.

This page was produced on \today.

The following is an ellipsis example.

If nominated \ldots, I will not serve.

\subsubsection*{Emphasizing Text}

Emphasized text is typed like this: \emph{this is emphasized}.
Bold       text is typed like this: \textbf{this is bold}.

You can have \emph{empahsized text \emph{within} emphasized text} too.

\subsubsection*{Preventing Line Breaks}

Typesetters break lines between words when possible and split words only between syllables 
(inserting a hyphen at the break).

Sometimes a line should not be broken between or within words. Human typesetters recognize 
these situations, but \TeX bust be told about some of them.

Line breaking should be prevented at certain interword spaces.

Typing a $\sim$ (a tilde character) produces ordinary interword space at which \TeX\ will never break a line.

Below are some examples indicating when a $\sim$ should be used.

Mr.~Jones

Figure~7

(1)~gnats

U.~S.~Grant

from 1 to~10

The \textbackslash  mbox command tells \TeX\ to print its entire argument on the same line.

Doctor \mbox{Lamport}, I presume?

\subsubsection*{Footnotes}

Footnotes are produced with the footnote command having the text of the footnote as its argument.

Gnus\footnote{A gnu is a big animal.} can be quite a gnusance.

A \textbackslash footnote command cannot be used in the argument of most commands;
for example, it can't appear in the argument of an \textbackslash footnote command.

\subsubsection*{Formulas}

If you're writing a technical document, it's likely to contain mathematical formulas.
A formula appearing in the middle of a sentence is enclosed by \textbackslash ( 
and \textbackslash ) commands.

Any spaces that you type in the formula are ignored.

Does \( x + y \) always equal \(y+x\)?

\TeX\ regards a formula as a word, which may be broken across lines at certain points, 
and space before the \textbackslash (  or after the \textbackslash ) is treated as an 
ordinary interword separation.

Subscripts are produced by the \_ command and superscripts by the \^{} command.

\( a_{1} > x^{2n} / y^{2n} \)

These two commands can only be used inside a mathematical formula.

When used in a formula, the right-quote character ' produces a prime (\('\)),
two in a row produce a double prime.

\end{document}                 % The input file ends with this command.
